\section{Monotone Domains}

The main theorems of this paper rely on the following lemma which is analogous to \cite[Theorem 1]{preparata1981}, which concerns polygons in the plane. Although the result we give fails to be a complete characterization of monotonicity in a given direction, it is enough for the purposes of later arguments.

\begin{lemma}\label{lem:headphones}
  Let $\Omega \subseteq \mathbb{R}^n$ be a bounded domain with smooth boundary $\Gamma$, and let ${\nu \colon \Gamma \to \mathbb{RP}^n}$ be the projectivized Gauss map. If $\abs{\nu^{-1}\{[\theta]\}\hspace{-1.5pt}} = 2$ for some $[\theta] \in \mathbb{RP}^{n-1}$, then $\Omega$ is monotone with respect to $[\theta]$.
\end{lemma}
\begin{proof}
  Consider the projection $\pi \colon 
  \mathbb{R}^n \to \mathbb{R}$ defined by $\pi(x) = \langle x, \theta \rangle$ and let $h = \pi|_\Gamma$. At a critical point $p$ of $h$, we have 
  \begin{equation*}
    \mathrm{d}h_p(v) = \inner{v, \theta} = 0
  \end{equation*}
  for all $v \in \mathrm{T}_p \Gamma$, which occurs if and only if $\nu(p) = [\theta]$. Since $\Gamma$ is compact and not contained in a line, $h$ attains a maximum and minimum at distinct points $p_{+}, p_{-} \in \Gamma$. By hypothesis, the preimage of $[\theta]$ under $\nu$ contains two elements, so there are no other critical values in $(\min h, \max h)$. By \cite[Theorem A]{yasuhiko2015}, $h^{-1}\{t\}$ is homeomorphic to $\mathbb{S}^{n-2}$ whenever $t$ is not an extreme value of $h$. Therefore, by the generalized Schoenflies theorem (see \cite{putman2025}), each slice $\Omega \cap \pi^{-1}\{t\}$ is homeomorphic to $\mathbb{B}^{n-1}$ whenever it is nonempty. The conclusion follows.
\end{proof}

We now prove the main theorems of the paper.

\begin{theorem}\label{thm:curvature}
    Let $\Omega \subseteq \mathbb{R}^n$ be a bounded domain with smooth boundary $\Gamma$. If the total absolute Gauss-Kronecker curvature satisfies $\int_{\Gamma} |K| \, \mathrm{d}\Gamma < 2\operatorname{vol}(\mathbb{S}^{n-1})$, then $\Omega$ is monotone with respect to some direction.
\end{theorem}

\begin{proof}
  Let $n \colon \Gamma \to \mathbb{S}^{n-1}$ be the Gauss map and $\nu = \rho \circ n$, where $\rho \colon \mathbb{S}^{n-1} \to \mathbb{RP}^{n-1}$ is the projection map. Since $\rho$ is a local isometry, the absolute Gauss-Kronecker curvature at a point $p \in \Gamma$ is given by the Jacobian
  \begin{equation}\label{eq:normal}
    |K(p)| = \sqrt{ \det (\d n_p^{*} \circ \d n_p)} = \sqrt{ \det (\d \nu_p^{*} \circ \d \nu_p)} = |\mathrm{J}_p\nu|.
  \end{equation}
  Define $\mu \colon \mathbb{RP}^{n-1} \to \mathbb{N}_{> 0}$ by $\mu([\theta]) = |\nu^{-1}\{[\theta]\}|$. Then by \eqref{eq:normal} and the smooth coarea formula \cite{chavel2006}, we have
  \begin{equation}
    \frac{1}{\operatorname{vol}(\mathbb{RP}^{n-1})} \int_{\mathbb{RP}^{n-1}} \mu \, \d \mathbb{RP}^{n-1} = \frac{1}{\frac{1}{2}\operatorname{vol}(\mathbb{S}^{n-1})}\int_{\Gamma} |K| \, \d \Gamma < 4,
  \end{equation}
  so the average multiplicity of a direction $[\theta] \in \mathbb{RP}^{n-1}$ is strictly bounded above by $4$. Note that $\deg_2 (\nu) =  0$ since $\nu$ factors through a double cover. It follows that $\mu$ takes on positive even values almost everywhere, so such an average is attained only if $\mu([\theta]) = 2$ for some $\theta \in \mathbb{S}^{n-1}$. The conclusion follows from Lemma \ref{lem:headphones}.
\end{proof}

By applying a standard smoothing argument, one may obtain an analogous result for polygons in the plane.

\begin{theorem}
Let $\Omega \subseteq \mathbb{R}^2$ be a domain with polygonal boundary $\Gamma$. If the sum of the absolute values of the exterior angles of $\Gamma$ is less than $4\pi$, then $\Omega$ is monotone with respect to some direction.
\end{theorem}

\begin{proof}
Let $\phi_1, \ldots, \phi_n$ be the exterior angles of $\Gamma$. By \say{rounding} each of the corners of $\Omega$, one may obtain a sequence $\Omega_i \to \Omega$ of domains with smooth boundaries $\Gamma_i$ such that the multiplicities $\mu_i \colon \mathbb{RP}^{1} \to \mathbb{N}_{> 0}$ of the projectivized Gauss maps do not vary with $i$, and
\begin{equation}
  \int_{\Gamma_i} |K| \, \d \Gamma_i = \sum_{k = 1}^{n} \,\abs{\phi_k}.
\end{equation}
It follows from the proof of Theorem \ref{thm:curvature} that there exists a consistent direction $[\theta]$ for which each $\Omega_i$ is monotone. Then for a line $\ell$ normal to $[\theta]$, the slices $\Omega_i \cap \ell$ are intervals which converge to $\Omega \cap \ell$. Since the limit of a sequence of intervals is also an interval and $\Omega \cap \ell$ is open in $\ell$, it must either be empty or homeomorphic to $\mathbb{B}^1$. The conclusion follows.
\end{proof}

As a corollary, one may show that every polygon with five or fewer sides is monotone in at least one direction.

\begin{corollary}
Let $\Omega \subseteq \mathbb{R}^2$ be a domain with $n$-sided polygonal boundary. If $n \leq 5$, then $\Omega$ is monotone with respect to some direction.
\end{corollary}
\begin{proof}
  Let $\theta_1, \ldots, \theta_n \in (-\pi, \pi)$ be the exterior angles of the boundary polygon. Suppose that $j$ angles are nonnegative and $k$ angles are negative. Since the sum of exterior angles is $2\pi$, we have
  \begin{equation}
    \sum_{i = 1}^{n} \abs{\hspace{1pt}\theta_i} = \sum_{\theta_i \geq 0} \theta_i -  \sum_{\theta_i < 0} \theta_i < 2\pi\min (j-1, k+1) \leq 4\pi,
  \end{equation}
  so $\Omega$ is monotone with respect to some direction.
\end{proof}

\begin{figure}[h]
\centering
\begin{tikzpicture}[scale=3]

% Coordinates of triangle
\coordinate (A) at (0,1);            
\coordinate (B) at (-0.866,-0.5);    
\coordinate (C) at (0.866,-0.5);     

% Centroid
\coordinate (G) at (0, 0); % point q

% Split vertices
\coordinate (A1) at (-0.03,1);
\coordinate (A2) at (0.03,1);

\coordinate (B1) at (-0.886,-0.50);
\coordinate (B2) at ($ (B)!0.15!(G) $);

\coordinate (C1) at (0.886,-0.50);
\coordinate (C2) at ($ (C)!0.15!(G) $);

% Hexagon points (snake around)
\coordinate (P1) at (A1);
\coordinate (P2) at (B1);
\coordinate (P3) at (C1);
\coordinate (P4) at (A2);
\coordinate (P5) at (C2);
\coordinate (P6) at (B2);

% Draw and fill hexagon
\fill[gray!30] (P1) -- (P2) -- (P3) -- (P4) -- (P5) -- (P6) -- cycle;
\draw[thick] (P1) -- (P2) -- (P3) -- (P4) -- (P5) -- (P6) -- cycle;

% ===== Add point q with label southeast =====
\node[fill=black, circle, inner sep=1.5pt, label=below right:$q$] at (G) {};

% ===== Red line through q, slope 1/5 =====
\draw[red, thick] ($ (G)+(-1,-0.2) $) -- ($ (G)+(1,0.2) $);

% ===== Blue solid vertical line through q to midpoint of A1-A2 =====
\draw[blue, thick] (0,-0.7) -- (0,1.2);

% ===== Blue dashed line: parallel translate to the left =====
\draw[blue, thick, dash pattern=on 3pt off 2pt] (-0.25,-0.7) -- (-0.25,1.2);

\draw[->, >=latex, blue, thick] (0,1.1) -- (-0.25,1.1);
\draw[->, >=latex, blue, thick] (0,-0.6) -- (-0.25,-0.6);


\end{tikzpicture}
\captionsetup{width=0.85\textwidth}
\caption{A hexagon which is not monotonic with respect to any direction. Lines through the centroid $q$ may be shifted to intersect the boundary in more than two points.}
\label{fig:hexagon}
\end{figure}


This result is sharp. For example, the hexagon in Figure \ref{fig:hexagon} is not monotone with respect to any direction. Indeed, any line through $q$ not passing through the \protect\say{slit} must intersect the boundary in four points. On the other hand, any line through $q$ passing through the slit may be translated left or right to achieve the same result.