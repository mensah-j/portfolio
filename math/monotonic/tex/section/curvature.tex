\section{Monotone Domains}

The main theorems of this paper rely on the following lemma which is analogous to \cite{}, which concerns polygons in the plane. However, the result we give fails to be a complete characterization of monotonicity in a given direction, though it is enough for our purposes.

\begin{proposition}
  Let $\Omega \subseteq \mathbb{R}^n$ be a bounded domain with smooth boundary $\Gamma$, and let ${\nu \colon \Gamma \to \mathbb{RP}^n}$ be the projectivized Gauss map. If $\abs{\nu^{-1}\{[\theta]\}\hspace{-1.5pt}} = 2$ for some $[\theta] \in \mathbb{RP}^{n-1}$, then $\Omega$ is monotone with respect to $[\theta]$.
\end{proposition}
\begin{proof}
  Consider the projection $\pi \colon 
  \mathbb{R}^n \to \mathbb{R}$ defined by $\pi(x) = \langle x, \theta \rangle$ and let $h = \pi|_\Gamma$. At a critical point $p$ of $h$, we have 
  \begin{equation*}
    \mathrm{d}h_p(v) = \inner{v, \theta} = 0
  \end{equation*}
  for all $v \in \mathrm{T}_p \Gamma$, which occurs if and only if $\nu(p) = [\theta]$. Since $\Gamma$ is compact and not contained in a line, $h$ attains a maximum and minimum at distinct points $p_{+}, p_{-} \in \Gamma$. By hypothesis, the preimage of $[\theta]$ under $\nu$ contains two elements, so there are no other critical values in $(\min h, \max h)$. By \cite[Theorem A]{}, $h^{-1}\{t\}$ is homeomorphic to $\mathbb{S}^{n-2}$ whenever $t$ is not an extreme value of $h$. Therefore, by the generalized Schoenflies theorem (see \cite{}), each slice $\Omega \cap \pi^{-1}\{t\}$ is homeomorphic to $\mathbb{B}^{n-1}$ whenever it is nonempty. The conclusion follows.
\end{proof}

We now prove the main theorems of the paper.

\begin{theorem}
    Let $\Omega \subseteq \mathbb{R}^n$ be a bounded domain with smooth boundary $\Gamma$. If the total absolute Gauss curvature satisfies $\int_{\Gamma} |K| \, \mathrm{d}\Gamma < 2\operatorname{vol}(\mathbb{S}^{n-1})$, then $\Omega$ is monotone with respect to some direction.
\end{theorem}

\begin{proof}
  Let $n \colon \Gamma \to \mathbb{S}^{n-1}$ be the Gauss map and $\nu = \rho \circ n$, where $\rho \colon \mathbb{S}^{n-1} \to \mathbb{RP}^{n-1}$ is the projection map. Since $\rho$ is a local isometry, the absolute Gauss-Kronecker curvature at a point $p \in \Gamma$ is given by the normal Jacobian
  \begin{equation}
    |K(p)| = \sqrt{ \det (\d n_p^{*} \circ \d n_p)} = \sqrt{ \det (\d \nu_p^{*} \circ \d \nu_p)} = |\mathrm{J}_p\nu|.
  \end{equation}
  Define $\mu \colon \mathbb{RP}^{n-1} \to \mathbb{N}_{> 0}$ by $\mu([\theta]) = |\nu^{-1}\{[\theta]\}|$ and let $P \subseteq \mathbb{RP}^{n-1}$ be the set of regular values of $\nu$. Since $\rho$ is a double cover and $P$ has full measure by Sard's theorem, its volume is given by $\operatorname{vol}(P) =  \frac{1}{2}\operatorname{vol}(\mathbb{S}^{n-1})$. Then by \eqref{} and the smooth coarea formula \cite{}, we have
  \begin{equation}
    \frac{1}{\operatorname{vol}(P)} \int_{P} \mu \, \d P = \frac{1}{\frac{1}{2}\operatorname{vol}(\mathbb{S}^{n-1})}\int_{\Gamma} |K| \, \d \Gamma < 4.
  \end{equation}
  Thus, the average multiplicity of a direction $[\theta] \in \mathbb{RP}^{n-1}$ is strictly bounded above by $4$. Furthermore, $\deg_2 (\nu) =  0$ since it factors through a double cover. It follows that $\mu$ only takes on positive even values, so such an average can attained only if $\mu([\theta]) = 2$ for some $\theta \in \mathbb{S}^{n-1}$. The conclusion follows from Proposition \ref{}.
\end{proof}

By applying a standard smoothing argument, one may obtain an analogous result for polygons in the plane.

\begin{theorem}
Let $\Omega \subseteq \mathbb{R}^2$ be a domain with polygonal boundary $\Gamma$. If the sum of the absolute values of the exterior angles of $\Gamma$ is less than $4\pi$, then $\Omega$ is monotone with respect to some direction.
\end{theorem}

\begin{proof}
Let $\theta_1, \ldots, \theta_n$ be the exterior angles of $\Gamma$. By \say{rounding} each of the corners of $\Omega$, one may obtain a sequence $\Omega_i \to \Omega$ of domains with smooth boundaries $\Gamma_i$ such that the multiplicities of the projective Gauss maps $\mu_i \colon \mathbb{RP}^{1} \to \mathbb{N}_{> 0}$ remain constant, and
\begin{equation}
  \int_{\Gamma_i} |K| \, \d \Gamma_i = \sum_{k = 1}^{n} \theta_k.
\end{equation}
It follows from the proof of Theorem \ref{} that there exists a consistent direction $[\theta]$ for which each $\Omega_i$ is monotone. Thus, for a line $\ell$ normal to $[\theta]$, the slices $\Omega_i \cap \ell$ are intervals which converge to $\Omega \cap \ell$. Since the limit of a sequence of intervals is also an interval, and $\Omega \cap \ell$ is open in $\ell$, it must either be empty or homeomorphic to $\mathbb{B}^1$. The conclusion follows.
\end{proof}

As a corollary, we obtain a resolution of the question posed in \cite{} concerning monotone polygons. 
\begin{corollary}
Let $\Omega \subseteq \mathbb{R}^2$ be a domain with $n$-sided polygonal boundary. If $n \leq 5$, then $\Omega$ is monotone with respect to some direction.
\end{corollary}
\begin{proof}
  Let $\theta_1, \ldots, \theta_n \in (-\pi, \pi)$ be the exterior angles of the boundary polygon. Suppose that $j$ angles are nonnegative and $k$ angles are negative. Since the sum of exterior angles is $2\pi$, we have
  \begin{equation}
    \sum_{i = 1}^{n} \abs{\hspace{1pt}\theta_i} = \sum_{\theta_i \geq 0} \theta_i -  \sum_{\theta_i < 0} \theta_i < 2\pi\min (j-1, k+1) \leq 4\pi,
  \end{equation}
  so $\Omega$ is monotone with respect to some direction.
\end{proof}