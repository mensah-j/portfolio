\section{Universal Coefficient Theorems}



\subsection{Homology}

Let $R$ be a commutative ring and let $(C_\bullet, \partial_\bullet)$ be a chain complex of $R$-modules. One may add \say{coefficients} to the chain complex by tensoring each $C_n$ with a fixed $R$-module $M$. This yields a new chain complex
\begin{equation*}
    \cdots \to C_{n+1} \otimes_R M \xrightarrow{\partial_{n+1} \otimes\, 1_M} C_n \otimes_R M \xrightarrow{\partial_n \otimes\, 1_M} C_{n-1} \otimes_R M \to \cdots
\end{equation*}
of $R$-modules, denoted by $C_\bullet \otimes_R M$. Each chain $c \in C_n \otimes_R M$ may be written as an \say{$M$-linear} combination of chains
\begin{equation*}
    c = \sum_{i \in I} c_i \otimes m_i
\end{equation*}
where $c_i \in C_n$ and $m_i \in M$. Note that $M$ is not required to have a ring structure, so after adding coefficients one should still regard $C_n \otimes_R M$ as an $R$-module.  The purpose of the universal coefficient theorem for homology is to relate the homology of $C_\bullet \otimes_R M$ to the homology of $C_\bullet$.

Before stating the theorem, we make a few important notational remarks. Let $\phi \colon A \to B$  be a map of $R$-modules and let $B_0$ be a subspace containing the image of $\phi$.  Typically, one denotes the corestriction $\phi|^{B_0} \colon A \to B_0$ by the symbol \say{$\phi$} again. However, this abuse of notation can lead to confusion when dealing with tensor products: the map
\begin{align*}
    \phi|^{B_0} \otimes 1_M &\colon A \otimes_R M \to B_0 \otimes_R M
\end{align*}
is \emph{not} a corestriction of $\phi \otimes 1_M$, unless one can identify $B_0 \otimes_R M$ with a submodule of $B \otimes_R M$ via the inclusion map $\iota \colon B_0 \hookrightarrow B$. In general, this is not possible, since $- \otimes_R M$ fails to preserve monomorphisms. Thus, to avoid confusion, we explicitly denote corestrictions in the proceeding argument as follows: given a chain complex $(C_\bullet, \partial_\bullet)$ and a map $\phi \colon A \to C_{n+1}$ such that $\im \phi \subset \ker \partial_{n}$, let
\begin{equation*}
  \phi^{\mathrm{Z}} = \phi|^{\ker \partial_{n}} \colon A \to \ker \partial_{n}
\end{equation*}
be the corestriction of $\phi$ to the $n$-cycles in $C_\bullet$. 


\begin{theorem}[Universal Coefficient Theorem for Homology]
    Let $R$ be a PID, and let $(C_\bullet, \partial_\bullet)$ be a chain complex of free $R$-modules. Then for any $R$-module $M$, there exists a short exact sequence of $R$-modules 
    \begin{equation*}
        0 \to \mathrm{H}_n(C_\bullet) \otimes_R M \to \mathrm{H}_n(C_\bullet \otimes_R M) \to \mathrm{Tor}_1^R(\mathrm{H}_{n-1}(C_\bullet), M) \to 0
    \end{equation*}
    for each $n \in \mathbb{Z}$, which is natural in both $C_\bullet$ and $M$. Furthermore, this sequence splits, but not naturally.
\end{theorem}

\begin{proof}
Let $\iota \colon \ker \partial_n \hookrightarrow C_n$ be the inclusion map. Then $\partial_{n+1} = \iota \circ \partial_{n+1}^{\mathrm{Z}}$, which implies that 
\begin{equation*}
(\partial_{n+1} \otimes_R 1_M)^\mathrm{Z} = \widetilde{\iota} \circ (\partial^{\mathrm{Z}}_{n+1} \otimes_R 1_M)
\end{equation*}
where $\widetilde{\iota}$ is the corestriction of $\iota \otimes 1_M$ to $\ker (\partial_n \otimes 1_M)$.

Since $R$ is a PID, and $C_{n-1}$ is a free $R$-module, it follows that $\im \partial_n \subseteq C_{n-1}$ is also free. Thus, the short exact sequence
\begin{equation*}
    0 \to \ker \partial_n \xrightarrow{\iota} C_n \xrightarrow{\partial_n} \mathrm{im}\, \partial_n \to 0
\end{equation*}
splits, which implies $\iota \otimes 1_M$ is a split monomorphism. The bottom right object in the commutative triangle
\begin{equation*}
  \begin{tikzcd}
    {C_{n+1} \otimes_R M} && {\ker(\partial_n) \otimes_R M} \\
    \\
    {C_{n+1} \otimes_R M} && {C_n\otimes1_M}
    \arrow["{\partial_{n+1}^{\mathrm{Z}} \otimes1_M}", from=1-1, to=1-3]
    \arrow[equals, from=1-1, to=3-1]
    \arrow["{\iota \otimes 1_M}", from=1-3, to=3-3]
    \arrow["{\partial_{n+1}\otimes1_M}"', from=3-1, to=3-3]
  \end{tikzcd}
\end{equation*}
may be replaced with $\ker (\partial_n \otimes 1_M)$, since $\im (\partial_{n+1} \otimes 1_M) \subseteq \ker (\partial_n \otimes 1_M)$. Takin
\end{proof}

\subsection{Cohomology}

Let $R$ be a commutative ring and let $(C_\bullet, \partial_\bullet)$ be a chain complex of $R$-modules. The dual cochain complex $(C^\bullet, \delta^\bullet)$ is formed by taking the dual of each $C_n$, yielding the sequence of $R$-modules
\begin{equation*}
    \cdots \to \mathrm{Hom}_R(C_{n-1}, R) \xrightarrow{\delta^{n-1}} \mathrm{Hom}_R(C_n, R) \xrightarrow{\delta^n} \mathrm{Hom}_R(C_{n+1}, R) \to \cdots
\end{equation*}
where $\delta^n = \partial_{n+1}^*$ is the pullback along $\partial_{n+1}$. More generally, one may bake coefficients into the cochain complex by instead taking the $R$-module of homomorphisms $\mathrm{Hom}_R(C_n, M)$ for a fixed $R$-module $M$. This yields a new cochain complex
\begin{equation*}
    \cdots \to \mathrm{Hom}_R(C_{n-1}, M) \xrightarrow{\delta^{n-1}} \mathrm{Hom}_R(C_n, M) \xrightarrow{\delta^n} \mathrm{Hom}_R(C_{n+1}, M) \to \cdots
\end{equation*}
of $R$-modules, denoted by $\mathrm{Hom}_R(C_\bullet, M)$. Note that this is a fundamentally different process than \emph{adding} coefficients by tensoring the dual chain complex with $M$, since $\mathrm{Hom}_R(C_n, M)$ is not generally isomorphic to $\mathrm{Hom}_R(C_n, R) \otimes_R M$. The purpose of the universal coefficient theorem for cohomology is to relate the cohomology of $\mathrm{Hom}_R(C_\bullet, M)$ to the cohomology of the dual complex $C^\bullet$.

\begin{theorem}[Universal Coefficient Theorem for Cohomology]
    Let $R$ be a PID, and let $(C_\bullet, \partial_\bullet)$ be a chain complex of free $R$-modules. Then for any $R$-module $M$, there exists a short exact sequence of $R$-modules 
    \begin{equation*}
        0 \to \mathrm{Ext}_R^1(\mathrm{H}_{n-1}(C_\bullet), M) \to \mathrm{H}^n(\mathrm{Hom}_R(C_\bullet, M)) \to \mathrm{Hom}_R(\mathrm{H}_n(C_\bullet), M) \to 0
    \end{equation*}
    for each $n \in \mathbb{Z}$, which is natural in both $C_\bullet$ and $M$. Furthermore, this sequence splits, but not naturally.
\end{theorem}