\section{Universal Coefficient Theorems}



\subsection{Homology}

Let $R$ be a commutative ring and let $(C_\bullet, \partial_\bullet)$ be a chain complex of $R$-modules. One may add \say{coefficients} to the chain complex by tensoring each $C_n$ with a fixed $R$-module $M$. This yields a new chain complex
\begin{equation*}
    \cdots \to C_{n+1} \otimes_R M \xrightarrow{\partial_{n+1} \otimes\, 1_M} C_n \otimes_R M \xrightarrow{\partial_n \otimes\, 1_M} C_{n-1} \otimes_R M \to \cdots
\end{equation*}
of $R$-modules, denoted by $C_\bullet \otimes_R M$. Each chain $c \in C_n \otimes_R M$ may be written as an \say{$M$-linear} combination of chains
\begin{equation*}
    c = \sum_{i \in I} c_i \otimes m_i
\end{equation*}
where $c_i \in C_n$ and $m_i \in M$. Note that $M$ is not required to have a ring structure, so after adding coefficients one should still regard $C_n \otimes_R M$ as an $R$-module.  The purpose of the universal coefficient theorem for homology is to relate the homology of $C_\bullet \otimes_R M$ to the homology of $C_\bullet$. Note that there is a map
\begin{equation*}
    \delta' \colon \mathrm{H}_n(C_\bullet) \otimes_R M \to \mathrm{H}_n(C_\bullet \otimes_R M) \quad \quad ([c] \otimes m) \overset{\delta'}{\mapsto} [c \otimes m].
\end{equation*}
which should be thought of as an approximation of the structure of $\mathrm{H}_n(C_\bullet \otimes_R M)$; the universal coefficient theorem tells us how well this estimate works.

\begin{theorem}[Universal Coefficient Theorem for Homology]
    Let $R$ be a PID, and let $(C_\bullet, \partial_\bullet)$ be a chain complex of free $R$-modules. Then for any $R$-module $M$, there exists a short exact sequence of $R$-modules 
    \begin{equation*}\label{eq:uct-homology}
        0 \to \mathrm{H}_n(C_\bullet) \otimes_R M \overset{\delta'}{\to} \mathrm{H}_n(C_\bullet \otimes_R M) \to \mathrm{Tor}_1^R(\mathrm{H}_{n-1}(C_\bullet), M) \to 0 \tag{$\star$}
    \end{equation*}
    for each $n \in \mathbb{Z}$, which is natural in both $C_\bullet$ and $M$. Furthermore, this sequence splits, but not naturally.
\end{theorem}

\begin{proof}
Since $R$ is a PID, and $C_{n-1}$ is a free $R$-module, it follows that $\im \partial_n \subseteq C_{n-1}$ is also free. Thus, the short exact sequence of chain complexes
\begin{equation*}
    0 \to \ker \partial_\bullet \to C_\bullet \to \im \partial_\bullet \to 0
\end{equation*}
splits, where the subcomplexes $\ker \partial_\bullet$ and $\mathrm{im}\, \partial_\bullet$ are defined by taking kernels and images degree-wise. Since this sequence is split, we may apply $- \otimes_R M$ to obtain a new short exact sequence of chain complexes
\begin{equation*}
    0 \to \ker \partial_\bullet \otimes_R M \to C_\bullet \otimes_R M \to \im \partial_\bullet \otimes_R M \to 0.
\end{equation*}
For a chain complex with trivial boundary maps, the homology groups coincide with the chain groups in each degree, so the associated long exact sequence in homology may be written as

\begin{equation*}
    \begin{tikzcd}
	& \cdots & {\im \partial_{n+1} \otimes_R M} \\
	{\ker \partial_n \otimes_R M} & {\mathrm{H}_n(C_\bullet \otimes_R M)} & {\im \partial_{n} \otimes_R M} \\
	{\ker \partial_{n-1} \otimes_R M} & \cdots
	\arrow[from=1-2, to=1-3]
	\arrow["{\l\delta_{n+1}}"{description}, from=1-3, to=2-1]
	\arrow[from=2-1, to=2-2]
	\arrow[from=2-2, to=2-3]
	\arrow["{\delta_n}"{description}, from=2-3, to=3-1]
	\arrow[from=3-1, to=3-2]
\end{tikzcd}
\end{equation*}
where $\delta_n = \iota_n \otimes_R 1_M$ for the inclusion $\iota_n \colon \im \partial_n \hookrightarrow \ker \partial_{n-1}$. The weaving lemma yields short exact sequences
\begin{equation*}
    0 \to \coker (\iota_{n+1} \otimes_R 1_M) \xrightarrow{\delta'_{n+1}} \mathrm{H}_n(C_\bullet \otimes_R M) \xrightarrow{\partial'_n} \ker(\iota_{n} \otimes_R 1_M) \to 0.
\end{equation*}
where $\delta'_{n+1}$ and $\partial'_n$ are the maps induced by $\delta_{n+1}$ and $\partial_n$, respectively. Since the tensor product preserves cokernels, the cokernel may be expressed as
\begin{equation*}
    \coker (\iota_{n+1} \otimes_R 1_M) \cong \coker \iota_{n+1} \otimes_R M \cong \mathrm{H}_n(C_\bullet) \otimes_R M.
\end{equation*}
Under this identification, $\delta'_{n+1}$ takes $([c] \otimes m) \mapsto [c \otimes m]$. Since the failure of the tensor product to preserve kernels is measured by the first torsion group, the kernel may be expressed as
\begin{equation*}
    \ker(\iota_n \otimes_R 1_M) \cong \mathrm{Tor}_1^R(\coker \iota_n, M) \cong \mathrm{Tor}_1^R(\mathrm{H}_{n-1}(C_\bullet), M).
\end{equation*}
Substituting these isomorphisms into the short exact sequence above yields the desired result. Since the connecting homomorphism and isomorphisms used were all natural, the entire sequence is natural in both $C_\bullet$ and $M$.
\end{proof}

\begin{remark}
    In the above proof, we implicitly used the fact that $\mathrm{Tor}$ was defined by taking a projective resolution of the first argument when computing the kernel of $\iota_n \otimes 1_M$. Since $\mathrm{Tor}$ is symmetric, one may also prove the theorem by taking a free resolution of the second argument $M$ instead: given a free resolution
    \begin{equation*}
        0 \to F_1 \to F_0 \to M \to 0
    \end{equation*}
    tensoring with $C_\bullet$ and applying the weaving lemma to the associated long exact sequence also yields the desired result.
\end{remark}

\subsection{Cohomology}

Let $R$ be a commutative ring and let $(C_\bullet, \partial_\bullet)$ be a chain complex of $R$-modules. The dual cochain complex $(C^\bullet, \delta^\bullet)$ is formed by taking the dual of each $C_n$, yielding the sequence of $R$-modules
\begin{equation*}
    \cdots \to \mathrm{Hom}_R(C_{n-1}, R) \xrightarrow{d^{n-1}} \mathrm{Hom}_R(C_n, R) \xrightarrow{d^n} \mathrm{Hom}_R(C_{n+1}, R) \to \cdots
\end{equation*}
where $d^n = (\partial_{n+1})^*$ is the pullback along $\partial_{n+1}$. More generally, one may bake coefficients into the cochain complex by instead taking the $R$-module of homomorphisms $\mathrm{Hom}_R(C_n, M)$ for a fixed $R$-module $M$. This yields a new cochain complex
\begin{equation*}
    \cdots \to \mathrm{Hom}_R(C_{n-1}, M) \xrightarrow{d^{n-1}} \mathrm{Hom}_R(C_n, M) \xrightarrow{d^n} \mathrm{Hom}_R(C_{n+1}, M) \to \cdots
\end{equation*}
of $R$-modules, denoted by $\mathrm{Hom}_R(C_\bullet, M)$. Note that this is a fundamentally different process than \emph{adding} coefficients by tensoring the dual chain complex with $M$. The purpose of the universal coefficient theorem for cohomology is to relate the cohomology of $\mathrm{Hom}_R(C_\bullet, M)$ to the cohomology of the dual complex $C^\bullet$.

\begin{theorem}[Universal Coefficient Theorem for Cohomology]
    Let $R$ be a PID, and let $(C_\bullet, \partial_\bullet)$ be a chain complex of free $R$-modules. Then for any $R$-module $M$, there exists a short exact sequence of $R$-modules 
    \begin{equation*}
        0 \to \mathrm{Ext}_R^1(\mathrm{H}_{n-1}(C_\bullet), M) \to \mathrm{H}^n(\mathrm{Hom}_R(C_\bullet, M)) \to \mathrm{Hom}_R(\mathrm{H}_n(C_\bullet), M) \to 0
    \end{equation*}
    for each $n \in \mathbb{Z}$, which is natural in both $C_\bullet$ and $M$. Furthermore, this sequence splits, but not naturally.
\end{theorem}

\begin{proof}
\end{proof}